%!TEX root = std.tex
\setcounter{chapter}{23}
\rSec0[containers]{Containers library}

\rSec1[containers.general]{General}

\pnum
This Clause describes components that \Cpp{} programs may use to
organize collections of information.

\pnum
The following subclauses describe
container requirements,
and components for
sequence containers and
associative containers,
as summarized in
\tref{containers.summary}.

\begin{libsumtab}{Containers library summary}{containers.summary}
\ref{container.requirements} & Requirements                     &                           \\ \rowsep
\ref{sequences}              & Sequence containers              &
  \tcode{<array>}, \tcode{<deque>}, \tcode{<forward_list>},
  \tcode{<list>}, \tcode{<vector>} \\ \rowsep
\ref{associative}            & Associative containers           &
  \tcode{<map>}, \tcode{<set>}     \\ \rowsep
\ref{unord}                  & Unordered associative containers &
  \tcode{<unordered_map>}, \tcode{<unordered_set>}    \\ \rowsep
\ref{container.adaptors}     & Container adaptors               &
  \tcode{<queue>}, \tcode{<stack>}, \added{\tcode{<flat_map>}}     \\ \rowsep
\ref{views}                  & Views                            & \tcode{<span>} \\
\end{libsumtab}


\setcounter{section}{2}
\setcounter{subsection}{3}

\noindent\makebox[\linewidth]{\rule{\textwidth}{0.4pt}}

\rSec2[sequence.reqmts]{Sequence containers}

\pnum
A sequence container organizes a finite set of objects, all of the same type, into a strictly
linear arrangement. The library provides four basic kinds of sequence containers:
\tcode{vector}, \tcode{forward_list}, \tcode{list}, and \tcode{deque}. In addition,
\tcode{array} is provided as a sequence container which provides limited sequence operations
because it has a fixed number of elements. The library also provides container
adaptors that make it easy to construct abstract data types, such
as \tcode{stack}s, \tcode{queue}s, \added{\tcode{flat_map}s,
or \tcode{flat_multimap}s, }out of the basic sequence container kinds (or out
of other kinds of sequence containers).

\setcounter{subsection}{6}

\noindent\makebox[\linewidth]{\rule{\textwidth}{0.4pt}}

\rSec2[associative.reqmts]{Associative containers}

\pnum
Associative containers provide fast retrieval of data based on keys.
The library provides four basic kinds of associative containers:
\tcode{set},
\tcode{multiset},
\tcode{map}
and
\tcode{multimap}.\added{  The library also provides container adaptors that
make it easy to construct abstract data types, such as \tcode{flat_map}s
or \tcode{flat_multimap}s, out of the basic sequence container kinds (or out
of other program-defined sequence containers).}

\setcounter{chapter}{24}
\setcounter{section}{5}
\rSec1[container.adaptors]{Container adaptors}

\rSec2[container.adaptors.general]{In general}

\pnum
The headers \tcode{<queue>}\changed{ and}{,} \tcode{<stack>}\added{, and
\tcode{<flat_map>}} define the container adaptors \tcode{queue},
\tcode{priority_queue}\changed{ and}{,} \tcode{stack}\added{, and
\tcode{flat_map}, respectively}.

\pnum

\changed{The}{Each} container adaptor\changed{s each take a \tcode{Container} template parameter, and
each constructor takes a \tcode{Container} reference argument. This container
is copied into the \tcode{Container} member of each adaptor.}{takes one or
more template parameters named \tcode{Container}, \tcode{KeyContainer},
or \tcode{MappedContainer} that denote the types of containers that the
container adaptor adapts.  Each container adaptor has at least one constructor
that takes a reference argument to one or more such template parameters.  For
each constructor reference argument to a container \tcode{C}, the constructor
copies the container into the container adaptor.}  If \changed{the
container}{\tcode{C}} takes an allocator, then a compatible allocator may be
passed in to the adaptor's constructor. Otherwise, normal copy or move
construction is used for the container argument. \changed{The first template
parameter \tcode{T} of the container adaptors shall denote the same type
as \tcode{Container::value_type}.}{For the container adaptors that take a
single container template parameter \tcode{Container}, the first template
parameter \tcode{T} of the container adaptor shall denote the same type
as \tcode{Container::value_type}.}

\pnum
The container adaptors each take a \tcode{Container} template parameter\added{
or \tcode{KeyContainer} and \tcode{MappedContainer} template parameters}, and
each constructor takes a \tcode{Container} reference argument\added{
or \tcode{KeyContainer} and \tcode{MappedContainer} reference
arguments}. \changed{This}{Such} container\changed{ is}{s are} copied into
\removed{the \tcode{Container} member of }each adaptor. If \changed{the}{such
a} container takes an allocator, then a compatible allocator may be passed in
to the adaptor's constructor. Otherwise, normal copy or move construction is
used for\removed{ the} container argument\added{s}.  \changed{T}{Except
for \tcode{flat_map} and \tcode{flat_multimap}, t}he first template
parameter \tcode{T} of the container adaptors shall denote the same type
as \tcode{Container::value_type}.

\pnum
For container adaptors, no \tcode{swap} function throws an exception unless
that exception is thrown by the swap of the
adaptor's \tcode{Container}\added{, \tcode{KeyContainer}, \tcode{MappedContainer}}, or
\tcode{Compare} object (if any).

\pnum
A constructor template of a container adaptor
shall not participate in overload resolution
if it has an \tcode{InputIterator} template parameter and
a type that does not qualify as an input iterator is deduced for that parameter.

\begin{addedblock}
\pnum
For container adaptors that have them, the \tcode{insert}, \tcode{emplace},
and \tcode{erase} members affect the validity of iterators, references, and
pointers to the adaptor's container(s) in the same way that the containers'
respective \tcode{insert}, \tcode{emplace}, and \tcode{erase} members do.

[\textit{Example:} A call to \tcode{flat_map<Key, T>::insert} can invalidate all iterators to
the \tcode{flat_map}.]
\end{addedblock}

\pnum
A deduction guide for a container adaptor shall not participate in overload resolution if any of the following are true:
\begin{itemize}
\item It has an \tcode{InputIterator} template parameter and a type that does not qualify as an input iterator is deduced for that parameter.
\item It has a \tcode{Compare} template parameter and a type that qualifies as an allocator is deduced for that parameter.
\item It has a \tcode{Container}\added{, \tcode{KeyContainer}, or \tcode{MappedContainer}} template parameter and a type that qualifies as an allocator is deduced for that parameter.
\item It has no \tcode{Container} template parameter, and it has an \tcode{Allocator} template parameter, and a type that does not qualify as an allocator is deduced for that parameter.
\item It has both \tcode{Container} and \tcode{Allocator} template parameters, and \tcode{uses_allocator_v<Container, Allocator>} is \tcode{false}.
\item \added{It has both \tcode{KeyContainer} and \tcode{Allocator} template parameters, and \\
\tcode{uses_allocator_v<KeyContainer, Allocator>} is \tcode{false}.}
\item \added{It has both \tcode{MappedContainer} and \tcode{Allocator} template parameters, and \\
\tcode{uses_allocator_v<MappedContainer, Allocator>} is \tcode{false}.}
\end{itemize}

\pnum
The exposition-only alias template \exposid{iter-value-type} defined in
[sequences.general]\added{ and the exposition-only alias
templates \exposid{iter-key-type} and
\exposid{iter-value-type} defined in [associative.general]} may appear in
deduction guides for container adaptors.

\begin{addedblock}
\pnum
The following exposition-only alias templates may appear in deduction guides
for container adaptors:

\begin{codeblock}
  template<class Container>
    using @\placeholder{cont-key-type}@ =                                // \expos
      remove_const_t<typename Container::value_type::first_type>;
  template<class Container>
    using @\placeholder{cont-mapped-type}@ =                             // \expos
      typename Container::value_type::second_type;
\end{codeblock}
\end{addedblock}

\noindent\makebox[\linewidth]{\rule{\textwidth}{0.4pt}}

\setcounter{subsection}{3}
\begin{addedblock}
\rSec2[flatmap.syn]{Header \tcode{<flat_map>} synopsis}%
\indexhdr{flatmap}%

\begin{codeblock}
#include <initializer_list>
#include <compare>

namespace std {
  // \ref{flatmap}, class template \tcode{flat_map}
  template<class Key, class T, class Compare = less<Key>,
           class KeyContainer = vector<Key>, class MappedContainer = vector<T>>
    class flat_map;

  struct sorted_unique_t { explicit sorted_unique_t() = default; };
  inline constexpr sorted_unique_t sorted_unique {};

  template<class Key, class T, class Compare, class KeyContainer, class MappedContainer,
           class Predicate>
    size_t erase_if(flat_map<Key, T, Compare, KeyContainer, MappedContainer>& c,
                    Predicate pred);

  // \ref{flatmultimap}, class template \tcode{flat_multimap}
  template<class Key, class T, class Compare = less<Key>,
           class KeyContainer = vector<Key>, class MappedContainer = vector<T>>
    class flat_multimap;

  struct sorted_equivalent_t { explicit sorted_equivalent_t() = default; };
  inline constexpr sorted_equivalent_t sorted_equivalent {};

  template<class Key, class T, class Compare, class KeyContainer, class MappedContainer,
           class Predicate>
    size_t erase_if(flat_multimap<Key, T, Compare, KeyContainer, MappedContainer>& c,
                    Predicate pred);

  template <class Key, class T, class Compare, class KeyContainer, class MappedContainer,
            class Allocator>
    struct uses_allocator<flat_map<Key, T, Compare, KeyContainer, MappedContainer>,
                          Allocator>;

  template <class Key, class T, class Compare, class KeyContainer, class MappedContainer,
            class Allocator>
    struct uses_allocator<flat_multimap<Key, T, Compare, KeyContainer, MappedContainer>,
                          Allocator>;

}
\end{codeblock}
\end{addedblock}

\noindent\makebox[\linewidth]{\rule{\textwidth}{0.4pt}}

\setcounter{subsection}{7}
\begin{addedblock}
\rSec2[flatmap]{Class template \tcode{flat_map}}

\rSec3[flatmap.overview]{Overview}

\pnum
\indexlibrary{\idxcode{flatmap}}%
A \tcode{flat_map} is a container adaptor that provides an associative
container interface that supports unique keys (i.e., contains at most one of each
key value) and provides for fast retrieval of values of another type \tcode{T}
based on the keys. \tcode{flat_map} supports iterators that meet the
\oldconcept{InputIterator} requirements and model the \libconcept{random_access_iterator}
concept ([iterator.concept.random.access]).

\pnum
A \tcode{flat_map} satisfies all of the requirements for a container
([container.reqmts]) and for a reversible container ([container.rev.reqmts]),
plus the optional container requirements ([container.opt.reqmts]).  \tcode{flat_map}
satisfies the requirements of an associative container ([associative.reqmts]), except that:
\begin{itemize}
\item it does not meet the requirements related to node handles ([container.node]),
\item it does not meet the requirements related to iterator invalidation, and
\item the time complexity of the operations that insert or erase a single
element from the map is linear, including the ones that take an insertion
position iterator.
\end{itemize}
\begin{note}A \tcode{flat_map} does not meet the additional requirements of an
allocator-aware container ([container.alloc.reqmts]).\end{note}

\pnum
A \tcode{flat_map} also provides most operations described
in [associative.reqmts] for unique keys.  This means that a
\tcode{flat_map} supports the \tcode{a_uniq} operations
in [associative.reqmts], but not the \tcode{a_eq} operations.  For
a \tcode{flat_map<Key,T>} the \tcode{key_type} is \tcode{Key} and the
\tcode{value_type} is \tcode{pair<const Key,T>}.

\pnum
Descriptions are provided here only for operations on \tcode{flat_map} that
are not described in one of those sets of requirements or for operations where
there is additional semantic information.

\pnum
A \tcode{flat_map} maintains the following invariants:
\begin{itemize}
\item it contains the same number of keys and values;
\item the keys are sorted with respect to the comparison object;
\item and the value at offset \tcode{off} within the value container is the value associated with the key at offset \tcode{off} within the key container.
\end{itemize}

\pnum
If any member function in [flatmap.defn] exits via an exception the invariants
are restored.

\pnum
Any sequence container ([sequence.reqmts]) \tcode{C} supporting \oldconcept{RandomAccessIterator}
can be used to instantiate \tcode{flat_map}, as long
as \tcode{C::size()} and \tcode{C::max_size()} do not throw. In
particular, \tcode{vector} ([vector]) and \tcode{deque} ([deque]) can be
used.  \begin{note}\tcode{vector<bool>} is not a sequence container.\end{note}

\pnum
The program is ill-formed if \tcode{Key} is not the same type
as \tcode{KeyContainer::value_type} or
\tcode{T} is not the same type as \tcode{MappedContainer::value_type}.

\pnum
The effect of calling a constructor that takes both \tcode{key_container_type}
and \tcode{mapped_container_type} arguments with containers of different sizes is
undefined.

\pnum
The effect of calling a constructor that takes a \tcode{sorted_unique_t}
argument with a range that is not sorted with respect to \tcode{compare}, or
that contains equal elements, is undefined.

\rSec3[flatmap.defn]{Definition}

\begin{codeblock}
namespace std {
  template <class Key, class T, class Compare = less<Key>,
            class KeyContainer = vector<Key>, class MappedContainer = vector<T>>
  class flat_map {
  public:
    // types
    using key_type                  = Key;
    using mapped_type               = T;
    using value_type                = pair<const key_type, mapped_type>;
    using key_compare               = Compare;
    using reference                 = pair<const key_type&, mapped_type&>;
    using const_reference           = pair<const key_type&, const mapped_type&>;
    using size_type                 = size_t;
    using difference_type           = ptrdiff_t;
    using iterator                  = @\impdefx{type of \tcode{flat_map::iterator}}@; // see 24.2
    using const_iterator            = @\impdefx{type of \tcode{flat_map::const_iterator}}@; // see 24.2
    using reverse_iterator          = std::reverse_iterator<iterator>;
    using const_reverse_iterator    = std::reverse_iterator<const_iterator>;
    using key_container_type        = KeyContainer;
    using mapped_container_type     = MappedContainer;

    class value_compare {
      friend flat_map;
    private:
      key_compare comp;                           // \expos
      value_compare(key_compare c) : comp(c) { }  // \expos
    public:
      bool operator()(const_reference x, const_reference y) const {
        return comp(x.first, y.first);
      }
    };

    struct containers
    {
      key_container_type keys;
      mapped_container_type values;
    };

    // \ref{flatmap.cons}, construct/copy/destroy
    flat_map() : flat_map(key_compare()) { }

    flat_map(key_container_type key_cont, mapped_container_type mapped_cont);
    template <class Allocator>
      flat_map(const key_container_type& key_cont,
               const mapped_container_type& mapped_cont,
               const Allocator& a);
    template <ranges::input_range R>
      explicit flat_map(const R& range,
                        const key_compare& comp = key_compare())
        : flat_map(ranges::begin(range), ranges::end(range), comp) { }
    template <ranges::input_range R, class Allocator>
      flat_map(const R& range, const Allocator& a);

    flat_map(sorted_unique_t,
             key_container_type key_cont, mapped_container_type mapped_cont);
    template <class Allocator>
      flat_map(sorted_unique_t, const key_container_type& key_cont,
               const mapped_container_type& mapped_cont, const Allocator& a);
    template <ranges::input_range R>
      flat_map(sorted_unique_t s,
               const R& range,
               const key_compare& comp = key_compare())
        : flat_map(s, ranges::begin(range), ranges::end(range), comp) { }
    template <ranges::input_range R, class Allocator>
      flat_map(sorted_unique_t, const R& range, const Allocator& a);

    explicit flat_map(const key_compare& comp)
      : c(), compare(comp) { }
    template <class Allocator>
      flat_map(const key_compare& comp, const Allocator& a);
    template <class Allocator>
      explicit flat_map(const Allocator& a);

    template <class InputIterator>
      flat_map(InputIterator first, InputIterator last,
               const key_compare& comp = key_compare())
        : c(), compare(comp)
        { insert(first, last); }
    template <class InputIterator, class Allocator>
      flat_map(InputIterator first, InputIterator last,
               const key_compare& comp, const Allocator& a);
    template<@\placeholder{container-compatible-range}@<value_type> R, class Allocator>
      flat_map(from_range_t, const R& range, const key_compare& comp = key_compare(),
               const Allocator& a = Allocator())
        : flat_map(comp, a)
        { insert_range(std::forward<R>(range)); }
    template <class InputIterator, class Allocator>
      flat_map(InputIterator first, InputIterator last,
               const Allocator& a);
    template<@\placeholder{container-compatible-range}@<value_type> R, class Allocator>
      flat_map(from_range_t, const R& range, const Allocator& a)
        : flat_map(std::forward<R>(range), a) { }

    template <class InputIterator>
      flat_map(sorted_unique_t s, InputIterator first, InputIterator last,
               const key_compare& comp = key_compare())
        : c(), compare(comp)
        { insert(s, first, last); }
    template <class InputIterator, class Allocator>
      flat_map(sorted_unique_t, InputIterator first, InputIterator last,
               const key_compare& comp, const Allocator& a);
    template <class InputIterator, class Allocator>
      flat_map(sorted_unique_t, InputIterator first, InputIterator last,
               const Allocator& a);

    flat_map(initializer_list<value_type>&& il,
             const key_compare& comp = key_compare())
        : flat_map(il, comp) { }
    template <class Allocator>
      flat_map(initializer_list<value_type>&& il,
               const key_compare& comp, const Allocator& a);
    template <class Allocator>
      flat_map(initializer_list<value_type>&& il, const Allocator& a);

    flat_map(sorted_unique_t s, initializer_list<value_type>&& il,
             const key_compare& comp = key_compare())
        : flat_map(s, il, comp) { }
    template <class Allocator>
      flat_map(sorted_unique_t, initializer_list<value_type>&& il,
               const key_compare& comp, const Allocator& a);
    template <class Allocator>
      flat_map(sorted_unique_t, initializer_list<value_type>&& il,
               const Allocator& a);

    flat_map& operator=(initializer_list<value_type> il);

    // iterators
    iterator                begin() noexcept;
    const_iterator          begin() const noexcept;
    iterator                end() noexcept;
    const_iterator          end() const noexcept;

    reverse_iterator        rbegin() noexcept;
    const_reverse_iterator  rbegin() const noexcept;
    reverse_iterator        rend() noexcept;
    const_reverse_iterator  rend() const noexcept;

    const_iterator          cbegin() const noexcept;
    const_iterator          cend() const noexcept;
    const_reverse_iterator  crbegin() const noexcept;
    const_reverse_iterator  crend() const noexcept;

    // \ref{flatmap.capacity}, capacity
    [[nodiscard]] bool empty() const noexcept;
    size_type size() const noexcept;
    size_type max_size() const noexcept;

    // \ref{flatmap.access}, element access
    mapped_type& operator[](const key_type& x);
    mapped_type& operator[](key_type&& x);
    template<class K> mapped_type& operator[](K&& x);
    mapped_type& at(const key_type& x);
    const mapped_type& at(const key_type& x) const;
    template<class K> mapped_type& at(const K& x);
    template<class K> const mapped_type& at(const K& x) const;

    // \ref{flatmap.modifiers}, modifiers
    template <class... Args> pair<iterator, bool> emplace(Args&&... args);
    template <class... Args>
      iterator emplace_hint(const_iterator position, Args&&... args);

    pair<iterator, bool> insert(const value_type& x)
      { return emplace(x); }
    pair<iterator, bool> insert(value_type&& x)
      { return emplace(std::move(x)); }
    iterator insert(const_iterator position, const value_type& x)
      { return emplace_hint(position, x); }
    iterator insert(const_iterator position, value_type&& x)
      { return emplace_hint(position, std::move(x)); }

    template <class P> pair<iterator, bool> insert(P&& x);
    template <class P>
      iterator insert(const_iterator position, P&&);
    template <class InputIterator>
      void insert(InputIterator first, InputIterator last);
    template <class InputIterator>
      void insert(sorted_unique_t, InputIterator first, InputIterator last);
    template<@\placeholder{container-compatible-range}@<value_type> R>
      void insert_range(const R& range)
        { insert(ranges::begin(range), ranges::end(range)); }

    void insert(initializer_list<value_type> il)
      { insert(il.begin(), il.end()); }
    void insert(sorted_unique_t s, initializer_list<value_type> il)
      { insert(s, il.begin(), il.end()); }

    containers extract() &&;
    void replace(key_container_type&& key_cont, mapped_container_type&& mapped_cont);

    template <class... Args>
      pair<iterator, bool> try_emplace(const key_type& k, Args&&... args);
    template <class... Args>
      pair<iterator, bool> try_emplace(key_type&& k, Args&&... args);
    template<class K, class... Args>
      pair<iterator, bool> try_emplace(K&& k, Args&&... args);
    template <class... Args>
      iterator try_emplace(const_iterator hint, const key_type& k,
                           Args&&... args);
    template <class... Args>
      iterator try_emplace(const_iterator hint, key_type&& k, Args&&... args);
    template<class K, class... Args>
      iterator try_emplace(const_iterator hint, K&& k, Args&&... args);
    template <class M>
      pair<iterator, bool> insert_or_assign(const key_type& k, M&& obj);
    template <class M>
      pair<iterator, bool> insert_or_assign(key_type&& k, M&& obj);
    template<class K, class M>
      pair<iterator, bool> insert_or_assign(K&& k, M&& obj);
    template <class M>
      iterator insert_or_assign(const_iterator hint, const key_type& k,
                                M&& obj);
    template <class M>
      iterator insert_or_assign(const_iterator hint, key_type&& k, M&& obj);
    template<class K, class M>
      iterator insert_or_assign(const_iterator hint, K&& k, M&& obj);

    iterator erase(iterator position);
    iterator erase(const_iterator position);
    size_type erase(const key_type& x);
    template<class K> size_type erase(K&& x);
    iterator erase(const_iterator first, const_iterator last);

    void swap(flat_map& fm) noexcept;
    void clear() noexcept;

    // observers
    key_compare key_comp() const;
    value_compare value_comp() const;

    const key_container_type& keys() const noexcept      { return c.keys; }
    const mapped_container_type& values() const noexcept { return c.values; }

    // map operations
    iterator find(const key_type& x);
    const_iterator find(const key_type& x) const;
    template <class K> iterator find(const K& x);
    template <class K> const_iterator find(const K& x) const;

    size_type count(const key_type& x) const;
    template <class K> size_type count(const K& x) const;

    bool contains(const key_type& x) const;
    template <class K> bool contains(const K& x) const;

    iterator lower_bound(const key_type& x);
    const_iterator lower_bound(const key_type& x) const;
    template <class K> iterator lower_bound(const K& x);
    template <class K> const_iterator lower_bound(const K& x) const;

    iterator upper_bound(const key_type& x);
    const_iterator upper_bound(const key_type& x) const;
    template <class K> iterator upper_bound(const K& x);
    template <class K> const_iterator upper_bound(const K& x) const;

    pair<iterator, iterator> equal_range(const key_type& x);
    pair<const_iterator, const_iterator> equal_range(const key_type& x) const;
    template <class K>
      pair<iterator, iterator> equal_range(const K& x);
    template <class K>
      pair<const_iterator, const_iterator> equal_range(const K& x) const;

    friend bool operator==(const flat_map& x, const flat_map& y);

    friend @\placeholder{synth-three-way-result}@<value_type>
      operator<=>(const flat_map& x, const flat_map& y);

    friend void swap(flat_map& x, flat_map& y) noexcept
      { x.swap(y); }

  private:
    containers c;        // \expos
    key_compare compare; // \expos

    // \expos
    struct key_equiv {
      key_equiv(key_compare c) : comp(c) { }
      bool operator()(const_reference x, const_reference y) const {
        return !comp(x.first, y.first) && !comp(y.first, x.first);
      }
      key_compare comp;
    };
  };

  template <class R>
    flat_map(R)
      -> flat_map<@\placeholder{cont-key-type}@<R>, @\placeholder{cont-mapped-type}@<R>>;

  template <class KeyContainer, class MappedContainer>
    flat_map(KeyContainer, MappedContainer)
      -> flat_map<typename KeyContainer::value_type,
                  typename MappedContainer::value_type,
                  less<typename KeyContainer::value_type>,
                  KeyContainer, MappedContainer>;

  template <class KeyContainer, class MappedContainer, class Allocator>
    flat_map(KeyContainer, MappedContainer, Allocator)
      -> flat_map<typename KeyContainer::value_type,
                  typename MappedContainer::value_type,
                  less<typename KeyContainer::value_type>,
                  KeyContainer, MappedContainer>;

  template <class R>
    flat_map(sorted_unique_t, R)
      -> flat_map<@\placeholder{cont-key-type}@<R>, @\placeholder{cont-mapped-type}@<R>>;

  template <class KeyContainer, class MappedContainer>
    flat_map(sorted_unique_t, KeyContainer, MappedContainer)
      -> flat_map<typename KeyContainer::value_type,
                  typename MappedContainer::value_type,
                  less<typename KeyContainer::value_type>,
                  KeyContainer, MappedContainer>;

  template <class KeyContainer, class MappedContainer, class Allocator>
    flat_map(sorted_unique_t, KeyContainer, MappedContainer, Allocator)
      -> flat_map<typename KeyContainer::value_type,
                  typename MappedContainer::value_type,
                  less<typename KeyContainer::value_type>,
                  KeyContainer, MappedContainer>;

  template <class InputIterator, class Compare = less<@\placeholder{iter-key-type}@<InputIterator>>>
    flat_map(InputIterator, InputIterator, Compare = Compare())
      -> flat_map<@\placeholder{iter-key-type}@<InputIterator>, @\placeholder{iter-mapped-type}@<InputIterator>, Compare>;

  template<ranges::input_range R, class Compare = less<@\placeholder{range-key-type}@<R>>,
           class Allocator>
    flat_map(from_range_t, const R&, Compare = Compare(), Allocator = Allocator())
      -> flat_map<@\placeholder{range-key-type}@<R>, @\placeholder{range-mapped-type}@<R>, Compare, Allocator>;

  template <class InputIterator, class Compare = less<@\placeholder{iter-key-type}@<InputIterator>>>
    flat_map(sorted_unique_t, InputIterator, InputIterator, Compare = Compare())
      -> flat_map<@\placeholder{iter-key-type}@<InputIterator>, @\placeholder{iter-mapped-type}@<InputIterator>, Compare>;

  template<ranges::input_range R, class Allocator>
    flat_map(from_range_t, const R&, Allocator)
      -> flat_map<@\placeholder{range-key-type}@<R>, @\placeholder{range-mapped-type}@<R>>;

  template<class Key, class T, class Compare = less<Key>>
    flat_map(initializer_list<pair<Key, T>>, Compare = Compare())
      -> flat_map<Key, T, Compare>;

  template<class Key, class T, class Compare = less<Key>>
    flat_map(sorted_unique_t, initializer_list<pair<Key, T>>, Compare = Compare())
        -> flat_map<Key, T, Compare>;

  template <class Key, class T, class Compare, class KeyContainer, class MappedContainer,
            class Allocator>
    struct uses_allocator<flat_map<Key, T, Compare, KeyContainer, MappedContainer>,
                          Allocator>
      : bool_constant<uses_allocator_v<KeyContainer> && uses_allocator_v<MappedContainer>> {}
}
\end{codeblock}

\rSec3[flatmap.cons]{Constructors}

\indexlibrary{\idxcode{flatmap}!constructor}%
\begin{itemdecl}
flat_map(key_container_type key_cont, mapped_container_type mapped_cont);
\end{itemdecl}

\begin{itemdescr}
\pnum
\effects Initializes \tcode{c.keys} with \tcode{std::move(key_cont)} and
\tcode{c.values} with \tcode{std::move(mapped_cont)}; value-initializes
\tcode{compare}; sorts the range \range{begin()}{end()} with respect to
\tcode{value_comp()}; and finally erases the duplicate elements
as if by:
\begin{codeblock}
auto zv = ranges::zip_view(c.keys, c.values);
auto it = ranges::unique(zv, key_equiv(compare));
c.keys.erase(c.keys.begin() + distance(it, zv.end()), c.keys.end());
c.values.erase(c.values.begin() + distance(it, zv.end()), c.values.end());
\end{codeblock}

\pnum
\complexity
Linear in $N$ if the container arguments are already sorted with respect
to \tcode{value_comp()} and otherwise $N \log N$, where $N$
is \tcode{key_cont.size()}.
\end{itemdescr}

\indexlibrary{\idxcode{flatmap}!constructor}%
\begin{itemdecl}
flat_map(sorted_unique_t, key_container_type key_cont, mapped_container_type mapped_cont);
\end{itemdecl}

\begin{itemdescr}
\pnum
\effects Initializes \tcode{c.keys} with
\tcode{std::move(key_cont)} and \tcode{c.values} with
\tcode{std::move(mapped_cont)}; value-initializes \tcode{compare}.

\pnum
\complexity
Constant.
\end{itemdescr}

\indexlibrary{\idxcode{flatmap}!constructor}%
\begin{itemdecl}
template <class Allocator>
  flat_map(const key_container_type& key_cont,
           const mapped_container_type& mapped_cont,
           const Allocator& a);
template <ranges::input_range R, class Allocator>
  flat_map(const R& range, const Allocator& a);
template <class Allocator>
  flat_map(sorted_unique_t, const key_container_type& key_cont,
           const mapped_container_type& mapped_cont, const Allocator& a);
template <ranges::input_range R, class Allocator>
  flat_map(sorted_unique_t, const R& range, const Allocator& a);
template <class Allocator>
  flat_map(const key_compare& comp, const Allocator& a);
template <class Allocator>
  explicit flat_map(const Allocator& a);
template <class InputIterator, class Allocator>
  flat_map(InputIterator first, InputIterator last,
           const key_compare& comp, const Allocator& a);
template <class InputIterator, class Allocator>
  flat_map(InputIterator first, InputIterator last,
           const Allocator& a);
template <class InputIterator, class Allocator>
  flat_map(sorted_unique_t, InputIterator first, InputIterator last,
           const key_compare& comp, const Allocator& a);
template <class InputIterator, class Allocator>
  flat_map(sorted_unique_t, InputIterator first, InputIterator last,
           const Allocator& a);
template <class Allocator>
  flat_map(initializer_list<value_type>&& il,
           const key_compare& comp, const Allocator& a);
template <class Allocator>
  flat_map(initializer_list<value_type>&& il, const Allocator& a);
template <class Allocator>
  flat_map(sorted_unique_t, initializer_list<value_type>&& il,
           const key_compare& comp, const Allocator& a);
template <class Allocator>
  flat_map(sorted_unique_t, initializer_list<value_type>&& il,
           const Allocator& a);
\end{itemdecl}

\begin{itemdescr}
\pnum
\constraints \tcode{uses_allocator_v<key_container_type, Allocator> \&\&
uses_allocator_v<mapped_container_type, Allocator>} is \tcode{true}.

\pnum
\effects Equivalent to the corresponding non-allocator constructors except that \tcode{c.keys}
and \tcode{c.values} are constructed with uses-allocator construction
([allocator.uses.construction]).
\end{itemdescr}

\rSec3[flatmap.capacity]{Capacity}

\indexlibrarymember{size}{flatmap}%
\begin{itemdecl}
size_type size() const noexcept;
\end{itemdecl}

\begin{itemdescr}
\pnum
\returns \tcode{c.keys.size()}.
\end{itemdescr}

\indexlibrarymember{max_size}{flatmap}%
\begin{itemdecl}
size_type max_size() const noexcept;
\end{itemdecl}

\begin{itemdescr}
\pnum
\returns \tcode{min<size_type>(c.keys.max_size(), c.values.max_size())}.
\end{itemdescr}

\rSec3[flatmap.access]{Access}

\indexlibrary{\idxcode{operator[]}!\idxcode{flatmap}}%
\begin{itemdecl}
mapped_type& operator[](const key_type& x);
\end{itemdecl}

\begin{itemdescr}
\pnum
\effects
Equivalent to: \tcode{return try_emplace(x).first->second;}
\end{itemdescr}

\indexlibrary{\idxcode{operator[]}!\idxcode{flatmap}}%
\begin{itemdecl}
mapped_type& operator[](key_type&& x);
\end{itemdecl}

\begin{itemdescr}
\pnum
\effects
Equivalent to: \tcode{return try_emplace(std::move(x)).first->second;}
\end{itemdescr}

\indexlibrary{\idxcode{operator[]}!\idxcode{flatmap}}%
\begin{itemdecl}
template<class K> mapped_type& operator[](K&& x);
\end{itemdecl}

\begin{itemdescr}
\pnum
\constraints
The \grammarterm{qualified-id} \tcode{Compare::is_transparent} is valid and denotes a type.

\pnum
\effects
Equivalent to: \tcode{return try_emplace(std::forward<K>(x)).first->second;}
\end{itemdescr}

\indexlibrary{\idxcode{at}!\idxcode{flatmap}}%
\begin{itemdecl}
mapped_type&       at(const key_type& x);
const mapped_type& at(const key_type& x) const;
\end{itemdecl}

\begin{itemdescr}
\pnum
\returns
A reference to the \tcode{mapped_type} corresponding to \tcode{x} in \tcode{*this}.

\pnum
\throws
An exception object of type \tcode{out_of_range} if
no such element is present.

\pnum
\complexity Logarithmic.
\end{itemdescr}

\indexlibrary{\idxcode{at}!\idxcode{flatmap}}%
\begin{itemdecl}
template<class K> mapped_type&       at(const K& x);
template<class K> const mapped_type& at(const K& x) const;
\end{itemdecl}

\begin{itemdescr}
\pnum
\constraints The \grammarterm{qualified-id} \tcode{Compare::is_transparent} is valid and denotes a type.

\pnum
\expects The expression \tcode{find(x)} is well-formed and has well-defined behavior.

\pnum
\returns
A reference to the \tcode{mapped_type} corresponding to \tcode{x} in \tcode{*this}.

\pnum
\throws
An exception object of type \tcode{out_of_range} if
no such element is present.

\pnum
\complexity Logarithmic.
\end{itemdescr}

\rSec3[flatmap.modifiers]{Modifiers}

\indexlibrarymember{emplace}{flatmap}%
\begin{itemdecl}
template <class... Args> pair<iterator, bool> emplace(Args&&... args);
\end{itemdecl}

\begin{itemdescr}
\pnum \constraints \tcode{is_constructible_v<pair<key_type, mapped_type>, Arg\&\&...>} is \tcode{true}.

\pnum
\effects
Initializes an object \tcode{t} of type \tcode{pair<key_type, mapped_type>}
with \tcode{std::forward<Args>(args)...};  if the map already
contains an element whose key is equivalent to \tcode{t.first}, \tcode{*this}
is unchanged.  Otherwise, equivalent to:
\begin{codeblock}
auto key_it = ranges::upper_bound(c.keys, t.first, compare);
auto value_it = c.values.begin() + distance(c.keys.begin(), key_it);
c.keys.insert(key_it, std::move(t.first));
c.values.insert(value_it, std::move(t.second));
\end{codeblock}

\pnum
\returns
The \tcode{bool} component of the returned pair is \tcode{true} if and only if
the insertion took place, and the iterator component of the pair points to the
element with key equivalent to \tcode{t.first}.
\end{itemdescr}

\indexlibrarymember{insert}{flatmap}%
\begin{itemdecl}
template<class P> pair<iterator, bool> insert(P&& x);
template<class P> iterator insert(const_iterator position, P&& x);
\end{itemdecl}

\begin{itemdescr}
\pnum \constraints \tcode{is_constructible_v<pair<key_type, mapped_type>, P\&\&>} is \tcode{true}.

\pnum
\effects
The first form is equivalent to
\tcode{return emplace(std::forward<P>(x))}. The second form is
equivalent to \tcode{return emplace_hint(position, std::forward<P>(x))}.
\end{itemdescr}

\indexlibrarymember{insert}{flatmap}%
\begin{itemdecl}
template <class InputIterator>
  void insert(InputIterator first, InputIterator last);
\end{itemdecl}

\begin{itemdescr}
\pnum
\effects Adds elements to \tcode{c} as if by:
\begin{codeblock}
for (; first != last; ++first) {
  c.keys.insert(c.keys.end(), first->first);
  c.values.insert(c.values.end(), first->second);
}
\end{codeblock}
Then, sorts the range of newly inserted elements with respect
to \tcode{value_comp()}; merges the resulting sorted range and the sorted
range of pre-existing elements into a single sorted range; and finally erases
the duplicate elements as if by:
\begin{codeblock}
auto zv = ranges::zip_view(c.keys, c.values);
auto it = ranges::unique(zv, key_equiv(compare));
c.keys.erase(c.keys.begin() + distance(it, zv.end()), c.keys.end());
c.values.erase(c.values.begin() + distance(it, zv.end()), c.values.end());
\end{codeblock}

\pnum
\complexity
$N$ + $M \log M$, where $N$ is \tcode{size()} before the operation and $M$
is \tcode{distance(first, last)}.
\end{itemdescr}

\indexlibrarymember{insert}{flatmap}%
\begin{itemdecl}
template <class InputIterator>
  void insert(sorted_unique_t, InputIterator first, InputIterator last);
\end{itemdecl}

\begin{itemdescr}
\pnum
\effects Adds elements to \tcode{c} as if by:
\begin{codeblock}
for (; first != last; ++first) {
  c.keys.insert(c.keys.end(), first->first);
  c.values.insert(c.values.end(), first->second);
}
\end{codeblock}
Then, merges the sorted range of newly added elements and the sorted range of
pre-existing elements into a single sorted range; and finally
erases the duplicate elements as if by:
\begin{codeblock}
auto zv = ranges::zip_view(c.keys, c.values);
auto it = ranges::unique(zv, key_equiv(compare));
c.keys.erase(c.keys.begin() + distance(it, zv.end()), c.keys.end());
c.values.erase(c.values.begin() + distance(it, zv.end()), c.values.end());
\end{codeblock}

\pnum
\complexity
Linear in $N$, where $N$ is \tcode{size()} after the operation.
\end{itemdescr}

\indexlibrarymember{try_emplace}{flatmap}%
\begin{itemdecl}
template<class... Args>
  pair<iterator, bool> try_emplace(const key_type& k, Args&&... args);
template<class... Args>
  pair<iterator, bool> try_emplace(key_type&& k, Args&&... args);
template<class... Args>
  iterator try_emplace(const_iterator hint, const key_type& k, Args&&... args);
template<class... Args>
  iterator try_emplace(const_iterator hint, key_type&& k, Args&&... args);
\end{itemdecl}

\begin{itemdescr}
\pnum \constraints \tcode{is_constructible_v<mapped_type, Args\&\&...>} is \tcode{true}.

\pnum
\effects
If the map already contains an element whose key is equivalent to \tcode{k},
\tcode{*this} and \tcode{args...} are unchanged.  Otherwise equivalent to:
\begin{codeblock}
auto key_it = ranges::upper_bound(c.keys, k, compare);
auto value_it = c.values.begin() + distance(c.keys.begin(), key_it);
c.keys.insert(key_it, std::forward<decltype(k)>(k));
c.values.emplace(value_it, std::forward<Args>(args)...);
\end{codeblock}

\pnum
\returns
In the first two overloads, the \tcode{bool} component of the returned pair
is \tcode{true} if and only if the insertion took place.  The returned
iterator points to the map element whose key is equivalent to \tcode{k}.

\pnum
\complexity
The same as \tcode{emplace} for the first two overloads, and the same
as \tcode{emplace_hint} for the last two overloads.
\end{itemdescr}

\indexlibrarymember{try_emplace}{flatmap}%
\begin{itemdecl}
template<class K, class... Args>
  pair<iterator, bool> try_emplace(K&& k, Args&&... args);
template<class K, class... Args>
  iterator try_emplace(const_iterator hint, K&& k, Args&&... args);
\end{itemdecl}

\begin{itemdescr}
\pnum
\constraints
\begin{itemize}
\item The \grammarterm{qualified-id} \tcode{Compare::is_transparent} is valid and denotes a type.
\item \tcode{is_constructible_v<key_type, K\&\&>} is \tcode{true}.
\item \tcode{is_constructible_v<mapped_type, Args\&\&...>} is \tcode{true}.
\item For the first overload, \tcode{is_convertible_v<K\&\&, const_iterator>} and \tcode{is_convertible_v<K\&\&, iterator>} are both \tcode{false}.
\end{itemize}

\pnum
\expects The conversion from \tcode{k} into \tcode{key_type} constructs an
object \tcode{u}, for which \tcode{find(k) == find(u)} is true.

\pnum
\effects
If the map already contains an element whose key is equivalent to \tcode{k},
\tcode{*this} and \tcode{args...} are unchanged.  Otherwise equivalent to:
\begin{codeblock}
auto key_it = ranges::upper_bound(c.keys, k, compare);
auto value_it = c.values.begin() + distance(c.keys.begin(), key_it);
c.keys.emplace(key_it, std::forward<K>(k));
c.values.emplace(value_it, std::forward<Args>(args)...);
\end{codeblock}

\pnum
\returns
In the first overload, the \tcode{bool} component of the returned pair
is \tcode{true} if and only if the insertion took place.  The returned
iterator points to the map element whose key is equivalent to \tcode{k}.

\pnum
\complexity
The same as \tcode{emplace} and \tcode{emplace_hint}, respectively.
\end{itemdescr}

\indexlibrarymember{insert_or_assign}{flatmap}%
\begin{itemdecl}
template<class M>
  pair<iterator, bool> insert_or_assign(const key_type& k, M&& obj);
template<class M>
  pair<iterator, bool> insert_or_assign(key_type&& k, M&& obj);
template<class M>
  iterator insert_or_assign(const_iterator hint, const key_type& k, M&& obj);
template<class M>
  iterator insert_or_assign(const_iterator hint, key_type&& k, M&& obj);
\end{itemdecl}

\begin{itemdescr}
\pnum \constraints \tcode{is_assignable_v<mapped_type\&, M>} is \tcode{true} and
\tcode{is_constructible_v<mapped_type, M\&\&>} is \tcode{true}.

\pnum
\effects
If the map already contains an element \tcode{e} whose key is equivalent
to \tcode{k}, assigns \tcode{std::forward<M>(obj)} to \tcode{e.second}.
Otherwise equivalent to \tcode{try_emplace(std::forward<decltype(k)>(k),
std::forward<M>(obj))} for the first two overloads or
\tcode{try_emplace_hint(hint, std::forward<decltype(k)>(k), std::forward<M>(obj))} for the last two overloads.

\pnum
\returns
In the first two overloads, the \tcode{bool} component of the returned pair
is \tcode{true} if and only if the insertion took place.  The returned
iterator points to the map element whose key is equivalent to \tcode{k}.

\pnum
\complexity
The same as \tcode{emplace} for the first two overloads and the same
as \tcode{emplace_hint} for the last two overloads.
\end{itemdescr}

\indexlibrarymember{insert_or_assign}{flatmap}%
\begin{itemdecl}
template<class K, class M>
  pair<iterator, bool> insert_or_assign(K&& k, M&& obj);
template<class K, class M>
  iterator insert_or_assign(const_iterator hint, K&& k, M&& obj);
\end{itemdecl}

\begin{itemdescr}
\pnum
\constraints
\begin{itemize}
\item The \grammarterm{qualified-id} \tcode{Compare::is_transparent} is valid and denotes a type.
\item \tcode{is_constructible_v<key_type, K\&\&>} is \tcode{true}.
\item \tcode{is_assignable_v<mapped_type\&, M>} is \tcode{true}.
\item \tcode{is_constructible_v<mapped_type, M\&\&>} is \tcode{true}.
\end{itemize}

\pnum
\expects The conversion from \tcode{k} into \tcode{key_type} constructs an
object \tcode{u}, for which \tcode{find(k) == find(u)} is true.

\effects
If the map already contains an element \tcode{e} whose key is equivalent
to \tcode{k}, assigns \tcode{std::forward<M>(obj)} to \tcode{e.second}.
Otherwise equivalent to \tcode{try_emplace(std::forward<decltype(k)>(k),
std::forward<M>(obj))} for the first overload or
\tcode{try_emplace_hint(hint, std::forward<decltype(k)>(k), std::forward<M>(obj))}
for the second overload.

\pnum
\returns
In the first overload, the \tcode{bool} component of the returned pair
is \tcode{true} if and only if the insertion took place.  The returned
iterator points to the map element whose key is equivalent to \tcode{k}.

\pnum
\complexity
The same as \tcode{emplace} and \tcode{emplace_hint}, respectively.
\end{itemdescr}

\indexlibrarymember{swap}{flatmap}%
\begin{itemdecl}
void swap(flat_map& fm) noexcept;
\end{itemdecl}

\begin{itemdescr}
\pnum \effects Equivalent to:
\begin{codeblock}
ranges::swap(compare, fm.compare);
ranges::swap(c.keys, fm.c.keys);
ranges::swap(c.values, fm.c.values);
\end{codeblock}
\end{itemdescr}

\indexlibrarymember{extract}{flatmap}%
\begin{itemdecl}
containers extract() &&;
\end{itemdecl}

\begin{itemdescr}
\pnum \ensures \tcode{*this} is emptied, even if the function exits via exception.

\pnum \returns \tcode{std::move(c)}.
\end{itemdescr}

\indexlibrarymember{replace}{flatmap}%
\begin{itemdecl}
void replace(key_container_type&& key_cont, mapped_container_type&& mapped_cont);
\end{itemdecl}

\begin{itemdescr}
\pnum \expects
\tcode{key_cont.size() == mapped_cont.size()} is \tcode{true}, and the elements of
\tcode{key_cont} are sorted with respect to \tcode{compare}.

\pnum
\effects Equivalent to:
\begin{codeblock}
c.keys = std::move(key_cont);
c.values = std::move(mapped_cont);
\end{codeblock}
\end{itemdescr}

\rSec3[flatmap.erasure]{Erasure}

\indexlibrarymember{erase_if}{flatmap}%
\begin{itemdecl}
template <class Key, class T, class Compare, class KeyContainer, class MappedContainer,
          class Predicate>
  typename flat_map<Key, T, Compare, KeyContainer, MappedContainer>::size_type
    erase_if(flat_map<Key, T, Compare, KeyContainer, MappedContainer>& c,
             Predicate pred);
\end{itemdecl}

\begin{itemdescr}
\pnum
\effects
Removes the elements for which \tcode{pred} is \tcode{true}, as if by:
\begin{codeblock}
auto original_size = c.size();

auto [keys, values] = std::move(c).extract();

auto keys_out_it = keys.begin();
auto values_out_it = keys.begin();
auto values_it = keys.begin();
for (auto & key : keys) {
  auto & value = *values_it;
  if (!pred(pair<const Key&, const T&>(key, value))) {
    *keys_out_it++ = std::move(key);
    *values_out_it++ = std::move(value);
  }
  ++values_it;
}

keys.erase(keys_out_it, keys.end());
values.erase(values_out_it, values.end());

c.replace(std::move(keys), std::move(values));
return original_size - c.size();
\end{codeblock}
\end{itemdescr}

\rSec2[flatmultimap]{Class template \tcode{flat_multimap}}

\rSec3[flatmultimap.overview]{Overview}

\pnum
\indexlibrary{\idxcode{flatmultimap}}%
A \tcode{flat_multimap} is a container adaptor that provides an associative
container interface that supports equivalent keys (i.e., possibly containing
multiple copies of the same key value) and provides for fast retrieval of
values of another type \tcode{T} based on the keys. \tcode{flat_multimap}
supports iterators that meet the \oldconcept{InputIterator} requirements
and model the \libconcept{random_access_iterator} concept
([iterator.concept.random.access]).

\pnum
A \tcode{flat_multimap} satisfies all of the requirements for a container
([container.reqmts]) and for a reversible container ([container.rev.reqmts]),
plus the optional container requirements ([container.opt.reqmts]).  \tcode{flat_multimap}
satisfies the requirements of an associative container ([associative.reqmts]), except that:
\begin{itemize}
\item it does not meet the requirements related to node handles ([container.node]),
\item it does not meet the requirements related to iterator invalidation, and
\item the time complexity of the operations that insert or erase a single
element from the map is linear, including the ones that take an insertion
position iterator.
\end{itemize}
\begin{note}A \tcode{flat_multimap} does not meet the additional requirements of an
allocator-aware container ([container.alloc.reqmts]).\end{note}

\pnum
A \tcode{flat_multimap} also provides most operations described
in [associative.reqmts] for equal keys.  This means that a
\tcode{flat_multimap} supports the \tcode{a_eq} operations
in [associative.reqmts] but not the \tcode{a_uniq} operations.  For
a \tcode{flat_multimap<Key,T>} the \tcode{key_type} is \tcode{Key} and the
\tcode{value_type} is \tcode{pair<const Key,T>}.

\pnum
Except as otherwise noted, operations on \tcode{flat_multimap} are equivalent
to those of \tcode{flat_map}, except that \tcode{flat_multimap} operations do
not remove or replace elements with equal keys.  \begin{example}\tcode{flat_multimap}
constructors and emplace do not erase non-unique elements after sorting them.\end{example}

\pnum
A \tcode{flat_multimap} maintains the following invariants:
\begin{itemize}
\item it contains the same number of keys and values;
\item the keys are sorted with respect to the comparison object;
\item and the value at offset \tcode{off} within the value container is the value associated with the key at offset \tcode{off} within the key container.
\end{itemize}

\pnum
If any member function in [flatmultimap.defn] subclause exits via an exception
the invariants are restored.

\pnum
Any sequence container ([sequence.reqmts]) \tcode{C} supporting \oldconcept{RandomAccessIterator}
can be used to instantiate \tcode{flat_multimap}, as long
as \tcode{C::size()} and \tcode{C::max_size()} do not throw. In
particular, \tcode{vector} ([vector]) and \tcode{deque} ([deque]) can be
used.  \begin{note}\tcode{vector<bool>} is not a sequence container.\end{note}

\pnum
The program is ill-formed if \tcode{Key} is not the same type
as \tcode{KeyContainer::value_type} or
\tcode{T} is not the same type as \tcode{MappedContainer::value_type}.

\pnum
The effect of calling a constructor that takes both \tcode{key_container_type}
and \tcode{mapped_container_type} arguments with containers of different sizes is
undefined.

\pnum
The effect of calling a constructor that takes a \tcode{sorted_equivalent_t}
argument with a container or containers that are not sorted with respect
to \tcode{value_compare} is undefined.

\rSec3[flatmultimap.defn]{Definition}

\begin{codeblock}
namespace std {
  template <class Key, class T, class Compare = less<Key>,
            class KeyContainer = vector<Key>, class MappedContainer = vector<T>>
  class flat_multimap {
  public:
    // types
    using key_type                  = Key;
    using mapped_type               = T;
    using value_type                = pair<const key_type, mapped_type>;
    using key_compare               = Compare;
    using reference                 = pair<const key_type&, mapped_type&>;
    using const_reference           = pair<const key_type&, const mapped_type&>;
    using size_type                 = size_t;
    using difference_type           = ptrdiff_t;
    using iterator                  = @\impdefx{type of \tcode{flat_multimap::iterator}}@; // see 24.2
    using const_iterator            = @\impdefx{type of \tcode{flat_multimap::const_iterator}}@; // see 24.2
    using reverse_iterator          = std::reverse_iterator<iterator>;
    using const_reverse_iterator    = std::reverse_iterator<const_iterator>;
    using key_container_type        = KeyContainer;
    using mapped_container_type     = MappedContainer;

    class value_compare {
      friend flat_multimap;
    private:
      key_compare comp;                           // \expos
      value_compare(key_compare c) : comp(c) { }  // \expos
    public:
      bool operator()(const_reference x, const_reference y) const {
        return comp(x.first, y.first);
      }
    };

    struct containers
    {
      key_container_type keys;
      mapped_container_type values;
    };

    // \ref{flatmultimap.cons}, construct/copy/destroy
    flat_multimap() : flat_multimap(key_compare()) { }

    flat_multimap(key_container_type key_cont, mapped_container_type mapped_cont);
    template <class Allocator>
      flat_multimap(const key_container_type& key_cont,
                    const mapped_container_type& mapped_cont,
                    const Allocator& a);
    template <ranges::input_range R>
      explicit flat_multimap(const R& range,
                             const key_compare& comp = key_compare())
        : flat_multimap(ranges::begin(range), ranges::end(range), comp) { }
    template <ranges::input_range R, class Allocator>
      flat_multimap(const R& range, const Allocator& a);

    flat_multimap(sorted_equivalent_t,
                  key_container_type key_cont, mapped_container_type mapped_cont);
    template <class Allocator>
      flat_multimap(sorted_equivalent_t, const key_container_type& key_cont,
                    const mapped_container_type& mapped_cont, const Allocator& a);
    template <ranges::input_range R>
      flat_multimap(sorted_equivalent_t s,
                    const R& range,
                    const key_compare& comp = key_compare())
        : flat_multimap(s, ranges::begin(range), ranges::end(range), comp) { }
    template <ranges::input_range R, class Allocator>
      flat_multimap(sorted_equivalent_t, const R& range, const Allocator& a);

    explicit flat_multimap(const key_compare& comp)
      : c(), compare(comp) { }
    template <class Allocator>
      flat_multimap(const key_compare& comp, const Allocator& a);
    template <class Allocator>
      explicit flat_multimap(const Allocator& a);

    template <class InputIterator>
      flat_multimap(InputIterator first, InputIterator last,
                    const key_compare& comp = key_compare())
        : c(), compare(comp)
        { insert(first, last); }
    template <class InputIterator, class Allocator>
      flat_multimap(InputIterator first, InputIterator last,
                    const key_compare& comp, const Allocator& a);
    template<@\placeholder{container-compatible-range}@<value_type> R, class Allocator>
      flat_multimap(from_range_t, const R& range, const key_compare& comp = key_compare(),
                    const Allocator& a = Allocator())
        : flat_map(comp, a)
        { insert_range(std::forward<R>(range)); }
    template <class InputIterator, class Allocator>
      flat_multimap(InputIterator first, InputIterator last,
                    const Allocator& a);
    template<@\placeholder{container-compatible-range}@<value_type> R, class Allocator>
      flat_mutlimap(from_range_t, const R& range, const Allocator& a)
        : flat_map(std::forward<R>(range), a) { }

    template <class InputIterator>
      flat_multimap(sorted_equivalent_t, InputIterator first, InputIterator last,
                    const key_compare& comp = key_compare())
        : c(), compare(comp)
        { insert(s, first, last); }
    template <class InputIterator, class Allocator>
      flat_multimap(sorted_equivalent_t, InputIterator first, InputIterator last,
                    const key_compare& comp, const Allocator& a);
    template <class InputIterator, class Allocator>
      flat_multimap(sorted_equivalent_t, InputIterator first, InputIterator last,
                    const Allocator& a);

    flat_multimap(initializer_list<value_type>&& il,
                  const key_compare& comp = key_compare())
        : flat_multimap(il, comp) { }
    template <class Allocator>
      flat_multimap(initializer_list<value_type>&& il,
                    const key_compare& comp, const Allocator& a);
    template <class Allocator>
      flat_multimap(initializer_list<value_type>&& il, const Allocator& a);

    flat_multimap(sorted_equivalent_t s, initializer_list<value_type>&& il,
                  const key_compare& comp = key_compare())
        : flat_multimap(s, il, comp) { }
    template <class Allocator>
      flat_multimap(sorted_equivalent_t, initializer_list<value_type>&& il,
                    const key_compare& comp, const Allocator& a);
    template <class Allocator>
      flat_multimap(sorted_equivalent_t, initializer_list<value_type>&& il,
                    const Allocator& a);

    flat_multimap& operator=(initializer_list<value_type> il);

    // iterators
    iterator                begin() noexcept;
    const_iterator          begin() const noexcept;
    iterator                end() noexcept;
    const_iterator          end() const noexcept;

    reverse_iterator        rbegin() noexcept;
    const_reverse_iterator  rbegin() const noexcept;
    reverse_iterator        rend() noexcept;
    const_reverse_iterator  rend() const noexcept;

    const_iterator          cbegin() const noexcept;
    const_iterator          cend() const noexcept;
    const_reverse_iterator  crbegin() const noexcept;
    const_reverse_iterator  crend() const noexcept;

    // capacity
    [[nodiscard]] bool empty() const noexcept;
    size_type size() const noexcept;
    size_type max_size() const noexcept;

    // \ref{flatmultimap.modifiers}, modifiers
    template <class... Args> pair<iterator, bool> emplace(Args&&... args);
    template <class... Args>
      iterator emplace_hint(const_iterator position, Args&&... args);

    pair<iterator, bool> insert(const value_type& x)
      { return emplace(x); }
    pair<iterator, bool> insert(value_type&& x)
      { return emplace(std::move(x)); }
    iterator insert(const_iterator position, const value_type& x)
      { return emplace_hint(position, x); }
    iterator insert(const_iterator position, value_type&& x)
      { return emplace_hint(position, std::move(x)); }

    template <class P> pair<iterator, bool> insert(P&& x);
    template <class P>
      iterator insert(const_iterator position, P&&);
    template <class InputIterator>
      void insert(InputIterator first, InputIterator last);
    template <class InputIterator>
      void insert(sorted_equivalent_t, InputIterator first, InputIterator last);
    template<@\placeholder{container-compatible-range}@<value_type> R>
      void insert_range(const R& range)
        { insert(ranges::begin(range), ranges::end(range)); }

    void insert(initializer_list<value_type> il)
      { insert(il.begin(), il.end()); }
    void insert(sorted_unique_t s, initializer_list<value_type> il)
      { insert(s, il.begin(), il.end()); }

    containers extract() &&;
    void replace(key_container_type&& key_cont, mapped_container_type&& mapped_cont);

    iterator erase(iterator position);
    iterator erase(const_iterator position);
    size_type erase(const key_type& x);
    template<class K> size_type erase(K&& x);
    iterator erase(const_iterator first, const_iterator last);

    void swap(flat_multimap& fm) noexcept;
    void clear() noexcept;

    // observers
    key_compare key_comp() const;
    value_compare value_comp() const;

    const key_container_type& keys() const noexcept      { return c.keys; }
    const mapped_container_type& values() const noexcept { return c.values; }

    // map operations
    iterator find(const key_type& x);
    const_iterator find(const key_type& x) const;
    template <class K> iterator find(const K& x);
    template <class K> const_iterator find(const K& x) const;

    size_type count(const key_type& x) const;
    template <class K> size_type count(const K& x) const;

    bool contains(const key_type& x) const;
    template <class K> bool contains(const K& x) const;

    iterator lower_bound(const key_type& x);
    const_iterator lower_bound(const key_type& x) const;
    template <class K> iterator lower_bound(const K& x);
    template <class K> const_iterator lower_bound(const K& x) const;

    iterator upper_bound(const key_type& x);
    const_iterator upper_bound(const key_type& x) const;
    template <class K> iterator upper_bound(const K& x);
    template <class K> const_iterator upper_bound(const K& x) const;

    pair<iterator, iterator> equal_range(const key_type& x);
    pair<const_iterator, const_iterator> equal_range(const key_type& x) const;
    template <class K>
      pair<iterator, iterator> equal_range(const K& x);
    template <class K>
      pair<const_iterator, const_iterator> equal_range(const K& x) const;

    friend bool operator==(const flat_multimap& x, const flat_multimap& y);

    friend @\placeholder{synth-three-way-result}@<value_type>
      operator<=>(const flat_multimap& x, const flat_multimap& y);

    friend void swap(flat_multimap& x, flat_multimap& y) noexcept
      { x.swap(y); }

  private:
    containers c;        // \expos
    key_compare compare; // \expos
  };

  template <class R>
    flat_multimap(R)
      -> flat_multimap<@\placeholder{cont-key-type}@<R>, @\placeholder{cont-mapped-type}@<R>>;

  template <class KeyContainer, class MappedContainer>
    flat_multimap(KeyContainer, MappedContainer)
      -> flat_multimap<typename KeyContainer::value_type,
                       typename MappedContainer::value_type,
                       less<typename KeyContainer::value_type>,
                       KeyContainer, MappedContainer>;

  template <class KeyContainer, class MappedContainer, class Allocator>
    flat_multimap(KeyContainer, MappedContainer, Allocator)
      -> flat_multimap<typename KeyContainer::value_type,
                       typename MappedContainer::value_type,
                       less<typename KeyContainer::value_type>,
                       KeyContainer, MappedContainer>;

  template <class R>
    flat_multimap(sorted_equivalent_t, R)
      -> flat_multimap<@\placeholder{cont-key-type}@<R>, @\placeholder{cont-mapped-type}@<R>>;

  template <class KeyContainer, class MappedContainer>
    flat_multimap(sorted_equivalent_t, KeyContainer, MappedContainer)
      -> flat_multimap<typename KeyContainer::value_type,
                       typename MappedContainer::value_type,
                       less<typename KeyContainer::value_type>,
                       KeyContainer, MappedContainer>;

  template <class KeyContainer, class MappedContainer, class Allocator>
    flat_multimap(sorted_equivalent_t, KeyContainer, MappedContainer, Allocator)
      -> flat_multimap<typename KeyContainer::value_type,
                       typename MappedContainer::value_type,
                       less<typename KeyContainer::value_type>,
                       KeyContainer, MappedContainer>;

  template <class InputIterator, class Compare = less<@\placeholder{iter-key-type}@<InputIterator>>>
    flat_multimap(InputIterator, InputIterator, Compare = Compare())
      -> flat_multimap<@\placeholder{iter-key-type}@<InputIterator>, @\placeholder{iter-mapped-type}@<InputIterator>, Compare>;

  template<ranges::input_range R, class Compare = less<@\placeholder{range-key-type}@<R>>,
           class Allocator>
    flat_multimap(from_range_t, const R&, Compare = Compare(), Allocator = Allocator())
      -> flat_multimap<@\placeholder{range-key-type}@<R>, @\placeholder{range-mapped-type}@<R>, Compare, Allocator>;

  template <class InputIterator, class Compare = less<@\placeholder{iter-key-type}@<InputIterator>>>
    flat_multimap(sorted_equivalent_t, InputIterator, InputIterator,
                  Compare = Compare())
      -> flat_multimap<@\placeholder{iter-key-type}@<InputIterator>, @\placeholder{iter-mapped-type}@<InputIterator>, Compare>;

  template<ranges::input_range R, class Allocator>
    flat_multimap(from_range_t, const R&, Allocator)
      -> flat_multimap<@\placeholder{range-key-type}@<R>, @\placeholder{range-mapped-type}@<R>>;

  template<class Key, class T, class Compare = less<Key>>
    flat_multimap(initializer_list<pair<Key, T>>, Compare = Compare())
      -> flat_multimap<Key, T, Compare>;

  template<class Key, class T, class Compare = less<Key>>
    flat_multimap(sorted_equivalent_t, initializer_list<pair<Key, T>>,
                  Compare = Compare())
        -> flat_multimap<Key, T, Compare>;

  template <class Key, class T, class Compare, class KeyContainer, class MappedContainer,
            class Allocator>
    struct uses_allocator<flat_multimap<Key, T, Compare, KeyContainer, MappedContainer>,
                          Allocator>
      : bool_constant<uses_allocator_v<KeyContainer> && uses_allocator_v<MappedContainer>> {}
}
\end{codeblock}

\rSec3[flatmultimap.cons]{Constructors}

\indexlibrary{\idxcode{flatmultimap}!constructor}%
\begin{itemdecl}
flat_multimap(key_container_type key_cont, mapped_container_type mapped_cont);
\end{itemdecl}

\begin{itemdescr}
\pnum
\effects Initializes \tcode{c.keys} with \tcode{std::move(key_cont)} and
\tcode{c.values} with \tcode{std::move(mapped_cont)}; value-initializes
\tcode{compare}; and sorts the range \range{begin()}{end()} with respect to
\tcode{value_comp()}.

\pnum
\complexity
Linear in $N$ if the container arguments are already sorted with respect
to \tcode{value_comp()} and otherwise $N \log N$, where $N$
is \tcode{key_cont.size()}.
\end{itemdescr}

\indexlibrary{\idxcode{flatmultimap}!constructor}%
\begin{itemdecl}
flat_multimap(sorted_equivalent_t, key_container_type key_cont,
              mapped_container_type mapped_cont);
\end{itemdecl}

\begin{itemdescr}
\pnum
\effects Initializes \tcode{c.keys} with
\tcode{std::move(key_cont)} and \tcode{c.values} with
\tcode{std::move(mapped_cont)}; value-initializes \tcode{compare}.

\pnum
\complexity
Constant.
\end{itemdescr}

\indexlibrary{\idxcode{flatmultimap}!constructor}%
\begin{itemdecl}
template <class Allocator>
  flat_multimap(const key_container_type& key_cont,
                const mapped_container_type& mapped_cont,
                const Allocator& a);
template <ranges::input_range R, class Allocator>
  flat_multimap(const R& range, const Allocator& a);
template <class Allocator>
  flat_multimap(sorted_equivalent_t, const key_container_type& key_cont,
                const mapped_container_type& mapped_cont, const Allocator& a);
template <ranges::input_range R, class Allocator>
  flat_multimap(sorted_equivalent_t, const R& range, const Allocator& a);
template <class Allocator>
  flat_multimap(const key_compare& comp, const Allocator& a);
template <class Allocator>
  explicit flat_multimap(const Allocator& a);
template <class InputIterator, class Allocator>
  flat_multimap(InputIterator first, InputIterator last,
                const key_compare& comp, const Allocator& a);
template <class InputIterator, class Allocator>
  flat_multimap(InputIterator first, InputIterator last,
                const Allocator& a);
template <class InputIterator, class Allocator>
  flat_multimap(sorted_equivalent_t, InputIterator first, InputIterator last,
                const key_compare& comp, const Allocator& a);
template <class InputIterator, class Allocator>
  flat_multimap(sorted_equivalent_t, InputIterator first, InputIterator last,
                const Allocator& a);
template <class Allocator>
  flat_multimap(initializer_list<value_type>&& il,
                const key_compare& comp, const Allocator& a);
template <class Allocator>
  flat_multimap(initializer_list<value_type>&& il, const Allocator& a);
template <class Allocator>
  flat_multimap(sorted_equivalent_t, initializer_list<value_type>&& il,
                const key_compare& comp, const Allocator& a);
template <class Allocator>
  flat_multimap(sorted_equivalent_t, initializer_list<value_type>&& il,
                const Allocator& a);
\end{itemdecl}

\begin{itemdescr}
\pnum
\constraints \tcode{uses_allocator_v<key_container_type, Allocator> \&\&
uses_allocator_v<mapped_container_type, Allocator>} is \tcode{true}.

\pnum
\effects Equivalent to the corresponding non-allocator constructors except that \tcode{c.keys}
and \tcode{c.values} are constructed with uses-allocator construction
([allocator.uses.construction]).
\end{itemdescr}

\rSec3[flatmultimap.erasure]{Erasure}

\indexlibrarymember{erase_if}{flatmultimap}%
\begin{itemdecl}
template <class Key, class T, class Compare, class KeyContainer, class MappedContainer,
          class Predicate>
  typename flat_multimap<Key, T, Compare, KeyContainer, MappedContainer>::size_type
    erase_if(flat_multimap<Key, T, Compare, KeyContainer, MappedContainer>& c,
             Predicate pred);
\end{itemdecl}

\begin{itemdescr}
\pnum
\effects
Removes the elements for which \tcode{pred} is \tcode{true}, as if by:
\begin{codeblock}
auto original_size = c.size();

auto [keys, values] = std::move(c).extract();

auto keys_out_it = keys.begin();
auto values_out_it = keys.begin();
auto values_it = keys.begin();
for (auto & key : keys) {
  auto & value = *values_it;
  if (!pred(pair<const Key&, const T&>(key, value))) {
    *keys_out_it++ = std::move(key);
    *values_out_it++ = std::move(value);
  }
  ++values_it;
}

keys.erase(keys_out_it, keys.end());
values.erase(values_out_it, values.end());

c.replace(std::move(keys), std::move(values));
return original_size - c.size();
\end{codeblock}
\end{itemdescr}
\end{addedblock}

\noindent\makebox[\linewidth]{\rule{\textwidth}{0.4pt}}

Add to section [container.adaptors.format]:

\rSec2[container.adaptors.format]{Formatting}
\begin{addedblock}
\pnum
For each of \tcode{flat_map} and \tcode{flat_multimap} templates, the library provides the following formatter specialization where \placeholder{map-type} is the name of the template:
\begin{codeblock}
namespace std {
  template <class charT, formattable<charT> Key, formattable<charT> T, class... U>
  struct formatter<@\placeholder{map-type}@<Key, T, U...>, charT>
  {
  private:
    range_formatter<pair<const Key&, const T&>> underlying_; // \expos
  public:
    formatter();

    template <class ParseContext>
      constexpr typename ParseContext::iterator
        parse(ParseContext& ctx);

    template <class FormatContext>
      typename FormatContext::iterator
        format(const @\placeholder{map-type}@<Key, T, U...>& r, FormatContext& ctx) const;
  };
}
\end{codeblock}

\begin{itemdecl}
formatter();
\end{itemdecl}

\begin{itemdescr}
\pnum
\effects Equivalent to:

\begin{codeblock}
this->set_brackets(@\placeholder{STATICALLY-WIDEN}@<charT>("{"), @\placeholder{STATICALLY-WIDEN}@<charT>("}"));
this->underlying().set_brackets({}, {});
this->underlying().set_separator(@\placeholder{STATICALLY-WIDEN}@<charT>(": "));
\end{codeblock}
\end{itemdescr}

\begin{itemdecl}
template <class ParseContext>
  constexpr typename ParseContext::iterator
    parse(ParseContext& ctx);
\end{itemdecl}

\begin{itemdescr}
\pnum
\effects Equivalent to \tcode{return underlying_.parse(ctx);}
\end{itemdescr}

\begin{itemdecl}
template <class FormatContext>
  typename FormatContext::iterator
    format(const @\placeholder{map-type}@<Key, T, U...>& r, FormatContext& ctx) const;
\end{itemdecl}

\begin{itemdescr}
\pnum
\effects Equivalent to \tcode{return underlying_.format(r, ctx);}
\end{itemdescr}
\end{addedblock}
